\documentclass[12pt,a4paper]{report}
\usepackage{graphicx}
\addtolength{\oddsidemargin}{-.875in}
\addtolength{\evensidemargin}{-.875in}
\addtolength{\textwidth}{1.75in}
\addtolength{\topmargin}{-.875in}
\addtolength{\textheight}{1.75in}
\usepackage[T1]{fontenc}
\begin{document}
\begin{center}
\Huge\textbf{LAB 1}\\
\large{To study Sampling and Quantization of Image
}
\end{center}
\vspace{1.0cm}
\begin{enumerate}
\item\textbf{Sampling of the image
}
\begin{verbatim}
clc;
clear all;
close all;
p=imread('cameraman.tif');
[m n]=size(p);
p=double(p);

for i=1:2:m
   
    for j=1:2:n
       q=p(i,j);
       if((q>=0) && (q<=15))
           f(i,j)=15;
       elseif((q>15) && (q<=30))
           f(i,j)=30;
            elseif((q>30) && (q<=45))
           f(i,j)=45;
            elseif((q>45) && (q<=60))
           f(i,j)=60;
            elseif((q>60) && (q<=75))
           f(i,j)=75;
            elseif((q>75) && (q<=90))
           f(i,j)=90;
           elseif((q>90) && (q<=105))
           f(i,j)=105;
            elseif((q>105) && (q<=120))
           f(i,j)=120;
           elseif((q>120) && (q<=135))
           f(i,j)=135;
            elseif((q>135) && (q<=150))
           f(i,j)=150;
           elseif((q>150) && (q<=165))
           f(i,j)=165;
            elseif((q>165) && (q<=180))
           f(i,j)=180;
           elseif((q>180) && (q<=195))
           f(i,j)=195;
            elseif((q>195) && (q<=210))
           f(i,j)=210;
           elseif((q>210) && (q<=225))
           f(i,j)=225;
            elseif((q>225) && (q<=240))
           f(i,j)=240;
       else
           f(i,j)=255;
       end
    end
end
p=uint8(p);
imshow(p);
figure;
imhist(p);
figure;

f=uint8(f);
imshow(f);
figure;
imhist(f);
\end{verbatim}\vspace{0.5cm}
\item\textbf{Sampling of the Image
}
\begin{verbatim}
clc;
clear all;
close all;
p=imread('cameraman.tif');
[m n]=size(p);
u=1;

for i=1:2:m
    v=1;
    for j=1:2:n
       f(u,v)=p(i,j);
       v=v+1;
    end
     u=u+1;
end
imshow(p);
figure;
imshow(f);
\end{verbatim}\vspace{0.5cm}

\end{enumerate}
\pagebreak\begin{center}
\Huge\textbf{LAB 2}\\
\large{Spatial Domain Transformations
}
\end{center}
\vspace{1.0cm}
\begin{enumerate}
\item\textbf{To study Bit-Plane Slicing
}
\begin{verbatim}
clc;
clear all;
close all;
a=imread('bitplane.tif');
b=mod(a,2);
a=a/2;
b=b*128;
subplot(3,3,1);
imshow(a);
b=mod(a,2);
a=a/2;
b=b*64;
subplot(3,3,2);
imshow(a);
b=mod(a,2);
a=a/2;
b=b*32;
subplot(3,3,3);
imshow(a);
b=mod(a,2);
a=a/2;
b=b*16;
subplot(3,3,4);
imshow(a);
b=mod(a,2);
a=a/2;
b=b*8;
subplot(3,3,5);
imshow(a);
b=mod(a,2);
a=a/2;
b=b*4;
subplot(3,3,6);
imshow(a);
b=mod(a,2);
a=a/2;
b=b*2;
subplot(3,3,7);
imshow(a);
b=mod(a,2);
a=a/2;
b=b*1;
subplot(3,3,8);
imshow(a);
\end{verbatim}\vspace{0.5cm}
\item\textbf{To Study Contrast Streching
}
\begin{verbatim}
clc;
clear all;
close all;
b=imread('contraststretch.tif');
[m n]=size(b);
r1=input('enter r1: ');
r2=input('enter r2: ');
s1=input('enter s1: ');
s2=input('enter s2: ');
for i=1:m
    for j=1:n
         if((b(i,j)<r1))
        c(i,j)=s1/r1*(b(i,j));
        
         elseif((b(i,j)>=r1) && (b(i,j)<=r2))
        c(i,j)=((s2-s1)/(r2-r1)*(b(i,j)-r1))+s1;
        else
          c(i,j)=((s2-255)/(r2-255)*(b(i,j)-255))+255;   
        end
    end
end
imshow(b);
figure;
imshow(c);\end{verbatim}\vspace{0.5cm}
\item\textbf{Dont know what this code is about
}
\begin{verbatim}
b=imread('pout.tif');
t=input('start:');
y=input('end: ');
%q=input('scale: ');

[m n] = size(b);


for a=1:m
    for c=1:n
        e=b(a,c);
        if( (e>=t) && (e<=y))
                
               p(a,c)=150;
                
        else
            p(a,c)=e;
        end
    end
    
end

imshow(b);
figure;
imshow(p);\end{verbatim}\vspace{0.5cm}
\item\textbf{Log Transformation
}
\begin{verbatim}
clc;
clear all;
close all;
b=imread('logtm.tif');
[m n]=size(b);
b=double(b);
for i=1:m
    for j=1:n
        c(i,j)=10*log(1+b(i,j));
    end
end
a=uint8(c);
imshow(b);
figure;
imshow(a);\end{verbatim}\vspace{0.5cm}
\item\textbf{Power Law Transformation
}
\begin{verbatim}
clc;
clear all;
close all;
b=imread('powerlaw.tif');
[m n]=size(b);
b=double(b);
for i=1:m
    for j=1:n
        c(i,j)=1*b(i,j)^(0.5);
    end
end
a=uint8(c);
d=uint8(b);
imshow(d);
figure;
imshow(a);\end{verbatim}\vspace{0.5cm}

\end{enumerate}
\pagebreak\begin{center}
\Huge\textbf{LAB 3}\\
\large{Histogram Processing
}
\end{center}
\vspace{1.0cm}
\begin{enumerate}
\item\textbf{Plot Histogram
}
\begin{verbatim}
b=imread('pout.tif');
[m n]=size(b);
for i=1:m
    for j=1:n
        a(i,j)=255-b(i,j);
    end
end
imshow(b);
figure;
imshow(a);
\end{verbatim}\vspace{0.5cm}
\item\textbf{Histogram Matching
}
\begin{verbatim}
b=imread('pout.tif');

[m,n] = size(b)
p=zeros(1,256);
%z=zeros(1,256);
for i=1:256
 %   z(i)=i;
for a=1:m
    for c=1:n
        if(b(a,c)==(i-1))
                
               p(i)=p(i)+1;
                
        end
    end
    
end
end
stem(p);\end{verbatim}\vspace{0.5cm}

\end{enumerate}
\pagebreak
\end{document}